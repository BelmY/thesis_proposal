%%%%%%%%%%%%%%%%%%%%%%%%%%%%%%%%%%%%%%%%%
% Thesis Proposal
%
% Francesc Wilhelmi Roca
% Boris Bellala, Cristina Cano, Anders Jonsson
% Wireless Networking Research Group
% Universitat Pompeu Fabra
%%%%%%%%%%%%%%%%%%%%%%%%%%%%%%%%%%%%%%%%%

\documentclass[12pt, a4paper,twoside]{article}
%\usepackage[latin1]{inputenc}
\usepackage[catalan,english]{babel}
\usepackage[cam,a4,center,frame]{crop}
\usepackage{graphicx}
\usepackage{times}
\usepackage{makeidx}
\usepackage{url}
\usepackage{subcaption}

%% Tikz libraries
\usepackage{tikz}
\usepackage{tkz-tab}
\usetikzlibrary{automata,arrows,positioning,calc}
\usetikzlibrary{shapes,snakes}

\usepackage{color}
\usepackage{amsmath}
\usepackage{mathtools}
\usepackage{fullpage}
\usepackage{algorithmic}
\DeclareMathOperator*{\argmin}{argmin}
\algsetup{linenosize=\small}
%\usepackage[table,xcdraw]{xcolor}
\usepackage{multirow}
\usepackage[super]{nth}
\usepackage{graphicx}
\usepackage{caption}
\usepackage[labelformat=simple]{subcaption}
\usepackage{setspace}
\usepackage{textcomp}
\usepackage{xspace}
\usepackage{siunitx}
\usepackage{epsfig}
\usepackage{epstopdf}
\usepackage{soul}
\usepackage{url}
\usepackage{tablefootnote}
\DeclareMathOperator{\E}{\mathbb{E}} % Expectation Symbol
\usepackage[linesnumbered,ruled]{algorithm2e}
\usepackage{booktabs}
\usepackage{pgfgantt}
\usepackage[catalan,english]{babel}
\pagestyle{plain}
\selectlanguage{english}

\title{Seminars Report}
\author{Francesc Wilhelmi}

\begin{document}

	\maketitle
	
	\selectlanguage{english}
	
	%%%%%%%%%%%%%%%%%%%%%%%%%%%%%%%%%%%%%%%
	% INTRODUCTION       								   %%%%%%%%%%%%%
	%%%%%%%%%%%%%%%%%%%%%%%%%%%%%%%%%%%%%%%
	\section{Introduction}
	\label{section:introduction}		
	The main purpose of this Thesis is to explore the applicability of Reinforcement Learning (RL) into the Spatial Reuse (SR) problem in Wireless Networks (WNs). Henceforth, it is sought the empowerment of WNs in order to let devices acting autonomously according to their requirements and possibilities. The matter of this Thesis is mainly motivated by the fact that the SR problem turns out to be NP-complete in dense scenarios (residential buildings, shopping malls, stadiums, trains, etc.). In addition, the variability in terms of user arrivals/departures and bandwidth requirements, adds an extra degree of complexity. Thus, real-time mechanisms turn out to be necessary to dynamically configure devices so as to adjust parameters such as the frequency channel, the transmit power or the sensitivity threshold.
		
	%%%%%%%%%%%%%%%%%%%%%%%%%%%%%%%%%%%%%%%
	% Career development       								   %%%%%%%%%%%%%
	%%%%%%%%%%%%%%%%%%%%%%%%%%%%%%%%%%%%%%%
	\section{Career development}
	\label{section:career}
	In order to be active in the research community, we have created the following profiles:
	\begin{itemize}
		\item Website: \url{https://www.upf.edu/web/fwilhelmi}
		\item Scholar: \url{https://scholar.google.es/citations?user=4EHXj4UAAAAJ&hl=ca&citsig=AMstHGShG4ofYlHpMmt1o34BA0qt08UFxQ}
		\item Orcid: \url{http://orcid.org/0000-0003-3936-535X}
	\end{itemize}		
	For a proper career development, we attempt to generate valuable and rigorous work in order to publish in relevant journals (e.g. Computer Networks\footnote{\url{https://www.journals.elsevier.com/computer-networks/}}, IEEE Intelligent Systems\footnote{\url{http://ieeexplore.ieee.org/xpl/RecentIssue.jsp?reload=true&punumber=9670}}, etc.). Furthermore, working with other researchers is fundamental to constantly improve and gain knowledge. Finally, another important aspect for a proper career development in research, is to be an active reviewer. Hence, during the first year we have participated in some reviews, even acting as TPC in the 2nd Workshop on Data Science for Internet of Things.\footnote{\url{https://ds-iot.org/}}
	
	%%%%%%%%%%%%%%%%%%%%%%%%%%%%%%%%%%%%%%%
	% General reproducibility aspects        								   %%%%%%%%%%%%%
	%%%%%%%%%%%%%%%%%%%%%%%%%%%%%%%%%%%%%%%
	\section{General reproducibility aspects }
	\label{section:reproducibility}		
	To properly share the generated knowledge, and with the aim of making reproducible the production derived from our research activity, we have created a Github account for the Wireless Networking Research Group.\footnote{\url{https://github.com/orgs/wn-upf/dashboard}} Furthermore, in relation to the research compendium provided by the Mar\'ia de Maeztu (MdM) excellence unit, we have established a set of guidelines for reproducibility:
	\begin{itemize}
		\item Publish the papers in arXiv for openness.
		\item Publish the code in Github. Furthermore, we follow the Clean Architecture, so that our code can be completely understood by a third party.
		\item Publish the experiments in Github together with the code. 
		\item Attach \textit{readme} files. In many occasions, it is required to provide some guidelines about how to run the code, as well as the accepted input parameters. In addition, it is worth to share other features of the simulation, such as the type of scenario used (including path-loss models, type of devices, traffic model, etc.).
	\end{itemize}				
	As an example, for the conference publication submitted to PIMRC \cite{wilhelmi2017implications}, we provided the source code used for simulations, specifying the concrete commit for the provided results. In addition, the code was prepared in a such way that simulations can be easily executed to reproduce similar results\footnote{Results may slightly vary due to the randomness added to the proposed scenario} to the ones shown in the publication.
	
	%%%%%%%%%%%%%%%%%%%%%%%%%%%%%%%%%%%%%%%
	% Data Management & Ethical issues         %%%%%%%%%%%%%
	%%%%%%%%%%%%%%%%%%%%%%%%%%%%%%%%%%%%%%%
	\section{Data Management and Ethical issues}
	\label{section:data_mgm}	
	Most of the activities planned for this Thesis involve using analytical models and simulated data related to wireless communications. Therefore, personal data is not expected to be manipulated. However, as the WNRG is getting involved into a collaboration with FON\footnote{\url{www.fon.com}}, there is some data to be used that may indirectly tell something about someone. In this case, the enterprise is the entity that collected the data, so we are only being requested to firm confidentiality agreements and to act responsibly.
		
	%%%%%%%%%%%%%%%%%%%%%%%%%%%%%%%%%%%%%%%
	% Intellectual Property       								   %%%%%%%%%%%%%
	%%%%%%%%%%%%%%%%%%%%%%%%%%%%%%%%%%%%%%%
	\section{Intellectual Property}
	\label{section:intellectual}		
	As a general rule, the research knowledge generated is aimed to be completely open, so that it can be freely used by other researchers. For that purpose, the contents generated in each project are published in Github under the GNU license (see Figure \ref{fig:gnu_license}), which provides a large range of permissions in exchange for using the same license. 
	\begin{figure}[t!]
		\centering
		\epsfig{file=images/gnu_license.png, width=16cm}
		\caption{GNU License}
		\label{fig:gnu_license}
	\end{figure}	
	However, there are some other cases in which knowledge cannot be freely shared, so a more restrictive licensing must be applied to the generated research output. This is the case the collaboration with FON, which requires setting the scene before making any publication.	

	%%%%%%%%%%%%%%%%%%%%%%%%%%%%%%%%%%%%%%%
	% Computing needs       								   %%%%%%%%%%%%%
	%%%%%%%%%%%%%%%%%%%%%%%%%%%%%%%%%%%%%%%
	\section{Computing needs}
	\label{section:computing}		
	To the proper development of the experimental part of some of the ongoing projects involved in the Thesis, a strong computational power is required. A really useful available tool is the High Performance Computing (HPC) cluster\footnote{\url{http://hpc.dtic.upf.edu/}}, which is composed by 15 nodes that provide high-performance calculation support to the DTIC community. In order to make use of the HPC, we have spiked on running advanced jobs in parallel. Some of our latest results (\cite{wilhelmi2017implications}) have been obtained from HPC executions with the Matlab parallel toolbox. In addition, to bring the usage of the HPC to the WNRG, we have created specific instructive material\footnote{Material available in \url{https://drive.google.com/file/d/0BzOhuzdX0br5b3pWMEFUZ3JDVUk/view?usp=sharing}.} to fulfil the needs of the group. Thus, HPC is introduced so that it can be rapidly used for the practical purposes of the Wireless Networking Research Group.	
		
	%%%%%%%%%%%%%%%%%%%%%%%%%%%%%%%%%%%%%%%
	% Scientific and general dissemination   	    %%%%%%%%%%%%%
	%%%%%%%%%%%%%%%%%%%%%%%%%%%%%%%%%%%%%%%
	\section{Scientific and general disseminations}
	\label{section:dissemination}			
	The main activity performed so far regarding scientific dissemination is the participation in the 5th EITIC Doctoral Workshop\footnote{\url{https://www.upf.edu/web/etic_doctoral_workshop}}, which was held at UPF the last March, 5th 2017. In this event, we presented \cite{wilhelmi2017improving}, an introduction to the wireless networks coexistence issues and the techniques that we aim to study along the Thesis to provide a solution. The poster was prepared in order to be simple, readable and eye-catching, so that a diverse audience can comprehend the problem and get interested on it.
	
	In parallel to the main research activity, we have also participated in the following dissemination activities:
	\begin{itemize}
		\item Teaching staff in ``Descobrint l'Internet de les Coses a través d'Arduino": introductory course about the Internet of Things (IoT) through Arduino. Held at UPF (Campus Junior) from 10th to 14th July, 2017.
		\item Jury member in the ``XI Award to Research Project in Applied Engineering and Mathematics"\footnote{\url{https://www.upf.edu/en/web/etic/research_project_award}}.
	\end{itemize}
	Both activities aim to disseminate Engineering to teenagers in order to help them determine their academic career.
	
	%%%%%%%%%%%%%%%%%%%%%%%%%%%%%%%%%%%%%%%
	% Scientific and general dissemination   	    %%%%%%%%%%%%%
	%%%%%%%%%%%%%%%%%%%%%%%%%%%%%%%%%%%%%%%
	\section{Gender}
	\label{section:gender}	
	Regarding the unfortunate gender imbalance that also reaches the scientific field, it is very important to be conscious of the problem and act in consequence to favour equality. For that, the most obvious act that we are able to do in research is to cite papers indiscriminately of the authors gender, which seems very natural but for many people is not. To the matter of our Thesis, we find several women researchers that stand out, which work is not only worth to be mentioned, but a must: 
	\begin{itemize}
		\item Cristina Cano\footnote{\url{https://scholar.google.es/citations?user=3fpyaLkAAAAJ&hl=es}}
		\item Konstantina Papagiannaki\footnote{\url{https://scholar.google.es/citations?user=xuVXDXkAAAAJ&hl=ca}}
		\item Christina Thorpe\footnote{\url{https://scholar.google.es/citations?user=Wed8UJwAAAAJ&hl=es}}
		\item Setareh Maghsudi\footnote{\url{http://dblp.uni-trier.de/pers/hd/m/Maghsudi:Setareh}}
	\end{itemize}	
			
	%%%%%%%%%%%%%%%%%%%%%%%%%%%%%%%%%%%%%%%
	% BIBLIOGRAPHY					      				   %%%%%%%%%%%%%
	%%%%%%%%%%%%%%%%%%%%%%%%%%%%%%%%%%%%%%%	
	\bibliographystyle{unsrt}
	\bibliography{bib}

\end{document}